\documentclass[oneside,final,14pt]{extarticle}
\usepackage[utf8]{inputenc}
\usepackage[russian]{babel}
\usepackage{a4wide}
\usepackage{vmargin}
\usepackage{textcomp}
\usepackage{graphicx}
\newcommand*{\No}{\textnumero}
\setpapersize{A4}
\setmarginsrb{3cm}{2cm}{1cm}{2cm}{0pt}{0mm}{0pt}{13mm}
\usepackage{indentfirst}
\sloppy
\setcounter{page}{2}
\clubpenalty = 10000
\widowpenalty = 10000

\begin{document}

\thispagestyle{empty}

\begin{center}
\vspace{-3cm}

\includegraphics[width=0.5\textwidth]{msu.pdf}\\
{\scshape Московский государственный университет имени М.~В.~Ломоносова}\\
Факультет вычислительной математики и кибернетики\\
Кафедра алгоритмических языков

\vfill

{\LARGE Отчёт о выполнении задания практикума}

\vspace{1cm}

    {\Huge\bfseries <<Ассистент в бронировании>>}
\end{center}

\vspace{1cm}

\begin{flushright}
  \large
    \textit{Студент 325 группы}\\
  М.\,А.~Гулак\\
\end{flushright}

\vfill

\begin{center}
Москва, 2023
\end{center}

\newpage
\section{Постановка задачи}

Требуется реализовать программу, в диалоге с которой пользователь может
забронировать номер в гостинице. Для гостиницы известны вид номеров
и их количество, посуточная оплата, а также их текущая занятость на ближайшую неделю.
Вся эта информация о гостинице задается в текстовом файле и включает данные об уже
забронированных номерах гостиницы.

Исходный запрос пользователя на бронирование может быть определен частично
(например, не задан вид номера гостиницы), и в ходе диалога ему предлагаются возможные варианты бронирования и
уточняются все его детали (в случае ошибок ввода делаются подсказки). В конце диалога
выводится детальное описание произведенного бронирования.

Диалог может допускать возможность отмены или изменения бронирования. Также по
специальному запросу может быть выведена (для администрации гостиницы) вся
информация о забронированных на неделю номерах.

\subsection{Базовые требования}

\begin{enumerate}
    \item Загрузка внутренного представления отеля из файла.
    % \item 
\end{enumerate}

\section{Модули проекта}

Проект состоит из следующих модулей (пример):
\begin{itemize}
    \item \texttt{Book.hs}~--- реализация функций для диалога с пользователем и логики самого диалога;
    \item \texttt{TypesHandle.hs}~--- реализация функций для работы с новыми типами данных;
    \item \texttt{DataTypes.hs}~--- объявление основных типов;
    \item \texttt{Constants.hs}~--- константы.

\end{itemize}

В модуле \texttt{Config.h} описаны следующие константные значения:
\begin{itemize}
    \item \texttt{width} и \texttt{height}~--- размер игрового поля;
    \item \texttt{defaultSpeed}~--- скорость передвижения игрока по умолчанию;
    \item и т.\,д.
\end{itemize}

В модуле \texttt{Types.h} описаны следующие типы:
\begin{itemize}
    \item \texttt{GameWord}~--- описывает игровой мир, состоящий из ...
    \item \texttt{Player}~--- описывает игрока ...
    \item и т.\,д.
\end{itemize}

В модуле \texttt{Game.h} реализованы следующие функции:
\begin{itemize}
    \item \texttt{checkGameOver}~--- проверка окончания игры;
    \item \texttt{getWinner}~--- определение победителя;
    \item и т.\,д.
\end{itemize}


\section{Используемые библиотеки}

При реализации использовались следующие библиотеки:
\begin{itemize}
    \item \texttt{Data.Maybe}
    \item \texttt{Text.Read} 
\end{itemize}

\section{Сценарии работы с приложением}

В этой части может содержаться краткая справка для пользователя приложения,
например:
\begin{itemize}
    \item Запуск проекта, формат конфигурационных файлов, если есть и т.\,д.
    \item Описание управления (консольное приложение/графический интерфейс/управление
          мышью или с клавиатуры), взаимодействие с пользователем.
    \item Режимы игры, выбор уровня, поведение приложения после окончания игры и т.\,п.
\end{itemize}

Можно добавлять скриншоты.

\end{document}
